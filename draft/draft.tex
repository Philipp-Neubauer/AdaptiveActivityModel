\documentclass{article}\usepackage[]{graphicx}\usepackage[]{color}
%% maxwidth is the original width if it is less than linewidth
%% otherwise use linewidth (to make sure the graphics do not exceed the margin)
\makeatletter
\def\maxwidth{ %
  \ifdim\Gin@nat@width>\linewidth
    \linewidth
  \else
    \Gin@nat@width
  \fi
}
\makeatother

\definecolor{fgcolor}{rgb}{0.345, 0.345, 0.345}
\newcommand{\hlnum}[1]{\textcolor[rgb]{0.686,0.059,0.569}{#1}}%
\newcommand{\hlstr}[1]{\textcolor[rgb]{0.192,0.494,0.8}{#1}}%
\newcommand{\hlcom}[1]{\textcolor[rgb]{0.678,0.584,0.686}{\textit{#1}}}%
\newcommand{\hlopt}[1]{\textcolor[rgb]{0,0,0}{#1}}%
\newcommand{\hlstd}[1]{\textcolor[rgb]{0.345,0.345,0.345}{#1}}%
\newcommand{\hlkwa}[1]{\textcolor[rgb]{0.161,0.373,0.58}{\textbf{#1}}}%
\newcommand{\hlkwb}[1]{\textcolor[rgb]{0.69,0.353,0.396}{#1}}%
\newcommand{\hlkwc}[1]{\textcolor[rgb]{0.333,0.667,0.333}{#1}}%
\newcommand{\hlkwd}[1]{\textcolor[rgb]{0.737,0.353,0.396}{\textbf{#1}}}%

\usepackage{framed}
\makeatletter
\newenvironment{kframe}{%
 \def\at@end@of@kframe{}%
 \ifinner\ifhmode%
  \def\at@end@of@kframe{\end{minipage}}%
  \begin{minipage}{\columnwidth}%
 \fi\fi%
 \def\FrameCommand##1{\hskip\@totalleftmargin \hskip-\fboxsep
 \colorbox{shadecolor}{##1}\hskip-\fboxsep
     % There is no \\@totalrightmargin, so:
     \hskip-\linewidth \hskip-\@totalleftmargin \hskip\columnwidth}%
 \MakeFramed {\advance\hsize-\width
   \@totalleftmargin\z@ \linewidth\hsize
   \@setminipage}}%
 {\par\unskip\endMakeFramed%
 \at@end@of@kframe}
\makeatother

\definecolor{shadecolor}{rgb}{.97, .97, .97}
\definecolor{messagecolor}{rgb}{0, 0, 0}
\definecolor{warningcolor}{rgb}{1, 0, 1}
\definecolor{errorcolor}{rgb}{1, 0, 0}
\newenvironment{knitrout}{}{} % an empty environment to be redefined in TeX

\usepackage{alltt}
\usepackage{amsmath}
\usepackage{caption} 
\usepackage{array} 
\usepackage[backend=biber]{biblatex}
\addbibresource{model.bib}
\captionsetup[table]{singlelinecheck=false}

\title{Interactive effects of oxygen and temperature on ectotherms}
\author{Philipp Neubauer \and Ken H. Andersen}
\IfFileExists{upquote.sty}{\usepackage{upquote}}{}
\begin{document}

\maketitle

\section{Introduction}

Climate change in adapting ecosystems: 


QUESTIONS:
\begin{itemize}
\item Size-structure and adaptive foraging stabilise food-webs: is stability compromised by climate change? Does adaptive activity dampen effect of climate change?
\item How large are indirect effects relative to direct effects?
\item Are more even food-webs more resilient than food-webs with dominant species?
\end{itemize}

\section{Setting up the model}

\subsection{Key assumptions}
\begin{itemize}
\item Animals will adapt activity levels to optimise available energy
\item Growth is limited either by food capture, food processing capacity, or by available oxygen
\item Temperature acts directly on rates that are determined by enzymatic activity: digestive activity (via maximum consumption) and metabolic cost. The scaling with temperature (determined by the activation energy $E_a$) is assumed the same for these processes. Consequently, temperature only acts on ecological rates (food acquisition) via optimisation given temperature driven changes in enzymatic rates.


\end{itemize}

\subsection{Metabolically constrained activity model}

In the adaptive activity model, ectotherms adjust the relative amounts of time ($\tau$) spent foraging and resting to optimise the net energy/mass gain $P_C$. Since both energy gain and loss are sensitive to temperature and oxygen limitations, both the activity level and the net energy gain will be subject to these environmental constraints. Their interplay thus determines available energy for growth, reproduction, and, ultimately, organisms final size.

The model is written in terms of carbon (or energy) and oxygen balance equations:

\begin{align}
f_C &= \frac{\tau }{\tau  + \frac{h c_T w^{q-p}}{\gamma\Theta} } \label{eq:f} \\
P_C &= S_C - D_C \\
  &=(1-\beta-\phi)f_C h c_T w^q  -(1+\tau \delta)c_T k w^n \\ 
P_{0_2} &= S_{O_2} - D_{O_2} \\
        &= f_{O_2}w^n - \omega \left( \beta f_C h c_T w^q +(1+\tau \delta) c_T k w^n \right)
\end{align}


where $f$ is the feeding level ([0,1]), determined by the fraction of time spent foraging (or proportion of maximum attack rate) $\tau$, consumption rate $\gamma w^p \Theta$ (search rate $\gamma w^p$ times prey availability $\Theta$) and maximum consumption $h w^q$. In the following, we will refer to $\tau$ as the activity fraction for sake of generality. Maximum consumption, being determined by digestive (enzymatic) processes, is assumed to scale with temperature as $c_T = e^{E_a(T-T_0)/kTT_0}$. Available carbon $P_C$ is determined by supply ($S_C$) from prey consumption ($f_C h c_T w^q$), with $\beta \approx 0.15$ a loss due to specific dynamic action (SDA, or heat increment; the energy spent absorbing food), and $\phi \approx 0.25$ is the fraction of food excreted and egested. Metabolic costs ($D_C$) are those of standard metabolism ($k w^n$), as well as active metabolism (scaled in units of standard metabolism as $\delta k w^n$), with the activity fraction $\tau$ determining the fraction of time that the active metabolism cost applies.  

The oxygen budget determines the metabolic scope $S_{0_2}$. Metabolic scope is the difference between oxygen supply $S_{O_2}/w^n=f_{O_2}$, the amount of oxygen supplied per unit weight, and oxygen demand. Demand ($D_{O_2}$) is the sum of oxygen used for SDA ($\beta f_C h c_T w^q$) and catabolism ($[1+\tau \delta] c_T k w^n$), with $\omega$ determining amount of oxygen required per unit of metabolised carbon (note: oxygen is measured in the same units as the carbon. To find the actual amount of oxygen used multiply by the amount of oxygen used to respire one unit of carbon). 

The maximum oxygen consumption---usually called the maximum metabolic rate (MMR), or active metbolic rate---is the oxygen consumption during maximal activity level that can be sustained over some time. This level is determined by oxygen delivery to organs and muscles. At constant temperature, oxygen supply as a function of ambient oxygen is assumed, in line with experimental evidence, to follow a saturating function. We specify $P_{50}$ as the point where the MMR (or oxygen supply) has dropped by 50\% relative to the saturation level $l$ \footnote{Could also use the same functional form as the consumption of food for simplicity: $f_{O_2} = \Gamma \Theta_{O_2}/\Gamma \Theta_{O_2} + h_{O_2} = \Theta_{O_2}/(\Theta_{O_2} + P_{50}).$}. Oxygen supply, and hence the MMR, are usually found to be temperature dependent (CITE), with a peak in MMR at an optimal temperature. We follow \cite{gnauck2013freshwater, lefrancois_influence_2003} and use a dome shaped function that peaks at the optimal temperature. Thus:

\begin{align}
f_{O_2}&=l(1-e^{O_2\log(0.5)/P50}),\\
l&=\zeta(\frac{T_{\text{max}}-T}{T_{\text{max}}-T_{\text{opt}}})^\eta \times \exp(-\eta\frac{T_{\text{max}}-T}{T_{\text{max}}-T_{\text{opt}}}),
\end{align}

with $f_{O_2}$ the oxygen supply (MMR), $T_{\text{max}}$ the lethal temperature for the species, $T_{\text{opt}}$ the temperature at which the MMR is optimised and $\eta$ determines the width of the dome-shape, and $\zeta$ its height (Figure \ref{fig:O2_fig}).



\begin{figure}
\includegraphics[width=\linewidth]{images/O2_fig-1} \caption[Maximum metabolic rate as a function of temperature, for three different optimal temperatures (panels---see panel label for optimal temperature in degrees celsius), and increasing values of ]{Maximum metabolic rate as a function of temperature, for three different optimal temperatures (panels---see panel label for optimal temperature in degrees celsius), and increasing values of $\eta$.}\label{fig:O2_fig}
\end{figure}



We now assume that organisms will adjust their activity level to maximise available energy under energetic and oxygen constraints. Energetically, the optimal activity level ($\tau_{\text{opt}}$) is found at $\frac{dP}{d\tau}=0$. We assume that the metabolic scope dictates the upper limit of this activity, such that at $\tau_{\text{max}}$, oxygen demand $D_{0_2}$ equals total supply $S_{0_2}$. Both temperature and oxygen will influence $\tau$, such that at a given temperature and oxygen concentration, $\tau_{T,O_2} = \text{min}\left( \tau_{\text{opt}},\tau_{\text{max}} \right)$, meaning we assume that animals will adapt their effort to optimise energy gain ($P_C^{\tau_{T,O_2}}$, the net production at $\tau_{T,O_2}$).

Assuming a constant investment $r$ in reproduction with weight ($D_{C_r}=rm$), the energy available for growth is directly proportional to $P_C^{\tau_{T,O_2}}$, and is thus a function of temperature and oxygen. Furthermore, organism size $m_{\infty}$, determined as the mass $m$ where $S_C^{\tau_{T,O_2}} = D_C^{\tau_{T,O_2}} + D_{C_r}^{\tau_{T,O_2}}$, is also determined by environmental factors.


\begin{table}
\caption{Parameters}
\begin{tabular}{ll}
Description & Value \\
\hline
Specific dynamic action & $\beta = 0.15$ \\
Egestion and excretion & $\phi = 0.25$ \\
Consumption rate$^1$ & $\Theta \gamma \approx h w^{p-q}$ (g$_C$/time) \\
Coef.~for maximum consumption rate & $h \approx 5 $g$_{WW}^{1-n}$/yr \\
Critical feeding level & $f_c \approx 0.1 $\\
Activity coefficient & $\delta \approx 1$ \\
Exponent for max.~consumption & $q=0.75$ \\
Exponent for clearance rate & $p=0.8$ \\
Exponent for std.~metabolism & $n=0.75$ \\
\hline
\multicolumn{2}{p{\textwidth}}{$^1$ This is under the assumption that the amount of encountered food $\Theta$ is roughly independent of body size.  This gives and order-of-magnitude estimate of the product $\Theta \gamma $.}
\end{tabular}
\end{table}

\subsection{Adaptive activity in marine food-webs}

In our model we assume that species regulate their activty level $\tau$ to optimise available energy for growth and reproduction. This concept aligns with the foraging arena concept that is often employed in food web models. In foraging arena theory, fish transition between vulnerable states (the foraging arena) and states of lesser vulnerability. The model outlined above does not yet consider that the optimisation of $\tau$ involves aspects of vulnerability associated with the foraging activity: the cost of foraging in the model above is purely metabolic. This is clearly not very realistic - optimisation of energy intake needs to be balanced against predation risk.

Predation risk at size can be derived by considering individuals of size $w$ in the context of a size-spectrum that models the abundance of individuals and species as a function of their wight (\cite{andersen_how_2009}). Given a size spectrum with slope $\lambda$, and assuming that all prey sizes are equally available to predation, predation mortality can be shown to be:

\begin{align}
\mu &= f_p \Phi w^{n-1},\\
\Phi &= f(\lambda,\eta(w,w_p))
\end{align}

where $f_p$, the predator feeding level, determines the predation pressure, and $\Phi$ is a constant that is dependent on slope of the size-spectrum $\lambda$ and the predator prey size selection function $\eta(w,w_p)$. We can assume that instead of optimising available energy, available energy is traded-off against predation risk, such that individuals adjust $\tau$ so as to optimise $F = P_C/\mu$. While conceptually straightforward, the adaptive activity now involves the predator feeding level which, according to equation \ref{eq:f}, is itself dependent on available prey. If we assume that prey are less vulnerable (or unavailable) when not feeding, then the predator feeding level depends on available prey through both prey abundance and $\tau_{prey}$. Thus, in the food-web context, the activity level (and availability) of each individual depends on the activity level and abundance of both their prey and predator - a phenomenon which creates indirect connections between the dynamics of all size-classes and species in the ecosystem (CITE Abrams 1992). 

To accommodate this complexity, we assume that optimisation in the ecosystem context is not instantaneous, but has some associated time-scale, and that the activity level adapts faster when the current behaviour is less optimal. This can be represented using a model for adaptive behaviour (CITE Matsuda et al 1996):

\begin{align}
\frac{\partial \tau}{\partial t} &= \kappa \tau(1-\tau) \frac{\partial F}{\partial \tau},
\label{eq:dtau}
\end{align}

where $\frac{\partial F}{\partial \tau}$ is the gradient of $F$ with respect to $\tau$, $\kappa$ determines the time-scale of behavioural adaptation. If the behaviour is entirely plastic, then $\kappa$ is likely high relative to the dynamics of the population dynamics of predators and prey. The factor of $\tau(1-\tau)$ restricts $\tau$ to be between 0 and 1, but also suggests that extreme activity levels are harder to change than intermediate activity levels.

This adaptive activity model can be integrated into existing size-spectrum models (e.g., CITE Hartvig) using the analytical solution to (\ref{eq:dtau}):

\begin{align}
\tau_{t+\Delta t} &= \frac{\tau_t e^{\Delta t \kappa \frac{\partial F}{\partial \tau}}}{\tau_t(e^{\Delta t \kappa \frac{\partial F}{\partial \tau}} - 1) + 1}
\end{align}

\printbibliography 
\end{document}
